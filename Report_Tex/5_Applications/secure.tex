\section{Secure Communication}
\textit{This section of the report has been worked upon by Kethari Narasimha Vardhan.}


\subsection{Introduction}
The current world entirely works digitally. It is very important to secure your data and communicate securely. Secure communication refers to the process of encrypting data exchanged between two parties to protect it from unauthorized access or interception. ECC can be used for secure communication in a number of ways. One of the most common ways is through the use of the Transport Layer Security (TLS) protocol and Secure Sockets Layer (SSL) protocol. \newline

\par
\noindent The main difference between SSL and TLS is that TLS provides better security and supports stronger encryption algorithms. The major role of Elliptic Curves in TLS is for key exchange and Digital Signatures. The latest version, SSL 3.0, was depreciated in 2015 due to serious vulnerabilities such as \textit{replay attacks} and \textit{man-in-the-middle attacks}. \newline

\par
\noindent Here, we see how ECC is used in TLS Protocol.

\subsection{Transport Layer Security (TLS)}

The two major uses of ECC in TLS are the key exchange and digital signatures. \newline

\par
\noindent\textbf{Key Exchange:} TLS uses a key exchange mechanism to establish a shared secret key between the client and server, which is used to encrypt and decrypt data exchanged between them. ECC can be used for the key exchange mechanism in TLS through the Elliptic Curve Diffie-Hellman (ECDH) key exchange protocol, which is a variant of the traditional Diffie-Hellman protocol that uses elliptic curves.

\noindent\textbf{Digital Signatures:} TLS also uses digital signatures to authenticate the identity of servers and clients and to protect against man-in-the-middle attacks. ECC can be used for digital signatures in TLS through the Elliptic Curve Digital Signature Algorithm (ECDSA), which is a variant of the Digital Signature Algorithm (DSA) that uses elliptic curves.

\newpage
\noindent For the \textbf{key exchange}, the algorithm of ECDH in TLS works as follows
\begin{enumerate}
    \item The client and server agree on a set of elliptic curve parameters, such as the curve type and size, to use for the key exchange.
    \item The server generates an ECC key pair consisting of a private key and a corresponding public key, and sends the public key to the client.
    \item The client generates its own ECC key pair, and uses the server's public key to compute a shared secret key.
    \item The client sends its own public key to the server, along with some additional information about the key exchange, encrypted using the shared secret key.
    \item The server uses its private key to decrypt the client's message and compute the shared secret key.
    \item Once the shared key has been established, it can be used to encrypt and decrypt data exchanged between the client and the server.
\end{enumerate}
\par
\noindent For the \textbf{digital signatures}, the algorithm of ECDSA in TLS works as follows
\begin{enumerate}
    \item The server generates an ECC key pair, consisting of a private key and a corresponding public key.
    \item The server sends its public key to a certificate authority (CA) to obtain a TLS certificate. The TLS certificate includes the server's public key and a digital signature generated using the CA's private key.
    \item When a client connects to the server, the server sends its TLS certificate to the client.
    \item The client uses the CA's public key to verify the digital signature in the TLS certificate and to authenticate the server's identity.
    \item Once the digital signature is verified by the client, it can be confident about the legitimacy and thus a secure connection has been established.
\end{enumerate}

\noindent The most common curves used SSL/TLS Protocols are NIST curves P-256 (also known as $secp256r1$) and P-384 (also known as $secp384r1$).

\newpage
\subsection{Example}
\noindent Let us take the standard example of the Wikipedia page, \textit{https://wikipedia.org/} \newline

\noindent The TLS certificate for this page is generally issued by Digicert Inc from the US. This Certificate uses Elliptic Curve Public Key Algorithm with Signature Algorithm being ECDSA. And the curve used is $secp256r1$, its parameters are;

\begin{itemize}
    \itemsep0em
    \item $y^2 = x^3 - 3x + b \:\: (\mathbb{F}_p)$
    \item $p = 2^{256} - 2^{224} + 2^{192} + 2^{96} - 1$
    \item $b = 2^{255} + 2^{119}$
\end{itemize}

% \noindent The algorithm used for signature is SHA-256 ECDSA, \textit{i.e.,} the Hashing used for signature is SHA-256. It is to be noted that, for every different client, different public-private keys are generated by the server. The key size, in this case, is 256 bits of ECC (which is equivalent to almost 2048-bit RSA). 
\noindent The major components in this TLS Certificate are
\begin{itemize}
    \itemsep0em
    \item Common Name (CN): This is the fully qualified domain name (FQDN) for which the certificate is issued. The CN must match the domain name that users enter in their web browser in order for the certificate to be trusted.\\ \textbf{E.g. *.wikipedia.org}
    \item Validity Period: This specifies the length of time that the certificate is valid. Most certificates are issued for a period of 1-2 years, after which they must be renewed. \textbf{E.g. Saturday, 18 November 2023 at 05:29:59 IST}
    \item Public Key: The certificate includes a public key that is used to encrypt data sent to the website. The private key, which is kept secret by the website owner, is used to decrypt the data. \textbf{E.g. Elliptic Curve Public Key (secp256r1)}
    \item Issuer: The certificate issuer is the organization that verifies the identity of the website owner and issues the certificate. The issuer's identity is also included in the certificate. \textbf{E.g. DigiCert Inc, USA}
    \item Signature Algorithm: This is the algorithm used to sign the certificate and verify its authenticity. Common signature algorithms include SHA-256 and SHA-384.\\ \textbf{E.g. ECDSA with SHA-384}
    \item Key Usage: This specifies how the public key in the certificate can be used, such as for encryption, digital signatures, or key agreement. \\ \textbf{E.g. Encrypt, Verify, and Derive}
\end{itemize}

\noindent \textit{Note: Some of the results in the components of the TLS certificate might vary as the server generates different public-private key pairs for different clients.}

