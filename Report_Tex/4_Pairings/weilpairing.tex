\chapter{Weil Pairing}\label{chap:weilpairing}

\section{Introduction}

Let $E/K$ be an elliptic curve over the field $K$, and $E[n]$ represents the group of torsion points of order $n$ in $E(\bar{K})$. 

If char$(K) \nmid n$, then $E[n] = Z/nZ \bigoplus Z/nZ$. This "two-dimensionality" of the group of torsion points is what makes pairings possible (there are no interesting pairings possible of cyclic groups). 

\section{Divisors}

For each point $P \in E(\bar{K})$, we define $[P]$ as the formal symbol $^{\cite{Silverman:1338326}}$.

\begin{definition} \label{defn:divisor}
A divisor $D$ on $E$ is a finite linear combination of such symbols with integer coefficients, i.e., $\alpha[P] + \beta[P] = (\alpha + \beta) [P]$. 
\end{definition}

The formal symbols generate a free Abelian group, generated by the formal symbols and divisors are elements in the group. Let this group of divisors be denoted by div$(E)$.

We define the following quantities
\begin{align*}
	&\text{deg} \Bigl( \sum_{j} a_j [P_j] \Bigr) = \sum_{j} a_j \in Z \\
	&\text{sum} \Bigl( \sum_{j} a_j [P_j] \Bigr) = \sum_{j} a_j P_j \in E(\bar{K})
\end{align*}

We will see that the divisors of degree $0$ (denoted by Div$^0 (E)$) form an important subgroup. 

There is a surjective homomorphism with the sum function
\begin{align*}
	&\text{sum : Div}^0 (E) \rightarrow E(\bar{K}) \\
	&\text{sum} ([P] - [\infty]) = P
\end{align*}

% Example in original document 

\begin{definition}
A function is said to have a \textbf{zero} at a point $P$ if it takes the value $0$ at $P$, and it has a \textbf{pole} at $P$ if it takes the value $\infty$ at $P$. 
\end{definition}

Let $P$ be a point. It can be shown that there is a function $u_P$, called a \textbf{uniformizer} at $P$, with $u(P) = 0$ and such that every function $f(x, y)$ can be written in the form 
\begin{equation*}
	f = u_P^r g, \quad \text{with } r \in Z \,\, \text{and } g(P) \neq 0, \infty
\end{equation*}

The existence of the uniformizer follows from the smoothness of the curve and showing that the partial derivatives are not all zero. At any finite point $P = (x_0, y_0)$ on an elliptic curve, the uniformizer $u_P$ can be taken from the equation of a line that passes through $P$ but is not tangent to $E$. A natural choice is $u_P := x - x_0 = 0$ when $y \neq 0$.

\begin{definition}
The \textbf{order} of $f$ at $P$ is given by ord$_P(f) = r$, where $f = u_p^r g$. 
\end{definition}
% Example in original document 

For the point $P = \infty$ and elliptic curve $E : y^2 = x^3 + Ax + B$, a uniformizer at $P$ is $u_{\infty} = \sfrac{x}{y}$.     

\begin{definition}
If $f$ is a function on $E$ that is not identically $0$, define the \textbf{divisor} of $f$ to be 
\begin{align}
	\text{div}(f) = \sum_{P \in E(\bar{K})}ord_P (f) [P] \in \text{Div}(E)
\end{align}
\end{definition}

\begin{theorem}
\label{zeros:poles}
Let $E$ be an elliptic curve and let $f$ be a function on $E$ that is not identically $0$. 
\begin{enumerate} 
	\item $f$ has only finitely many zeros and poles 
	\item $deg(div(f)) = 0$
	\item If $f$ has no zeros or poles (so $div(f) = 0$), then $f$ is a constant
\end{enumerate}
\end{theorem}

\begin{proof}
    Every rational function with coordinates from an algebraically closed field is of the form $\cfrac{f(X, Y)}{g(X, Y)}$ after homogenizing the function. Each term in the numerator and denominator have the same degree, which leads to the number of poles being equal to the number of zeros. 
\end{proof}

The divisor of a function is said to be a \textbf{principal divisor}. 

If we have a rational function $\frac{f(X, Y)}{g(X, Y)}$, then 
\begin{align*}
	\text{div} \left( \frac{f(X, Y)}{g(X, Y)} \right) = \text{div} (f(X, Y)) - \text{div} (g(X, Y))
\end{align*}

\begin{theorem}
\label{div_existence}
	Let E be an elliptic curve and $D$ be a divisor on $E$ that belongs to Div\,$^0(E)$. There is a function $f$ on $E$ with div($f$) = $D$ if and only if sum($D$) = $\infty$.
\end{theorem}

\begin{proof}

Suppose $P_1, P_2, P_3$ are three points on $E$ that lie on the line $ax + by + c = 0$. Then the function $f(x, y) = ax + by + c$ has zeros at $P_1, P_2, P_3$. The divisor of $f(x, y)$ is $div(ax + by + c) = [P_1] + [P_2] + [P_3] - 3[\infty]$. The line through the points $P_3, -P_3$ is $x - x_3 = 0$. So, we have $div(x-x_3) = [P_3] + [-P_3] - 2[\infty]$.

The sum $[P_1] + [P_2]$ can be replaced by $[P_1 + P_2] + [\infty] + div(g)$ from the above argument (where $P_1 + P_2 = P_3$) 

So, sum($div(g)) = P_1 + P_2 - (P_1 + P_2) - \infty = \infty$.

Let sum of all terms in $D$ with positive coefficients be $[P] + n_1 [\infty] + div(f_1)$ and let the sum of all terms in $D$ with positive coefficients be $[Q] + n_2 [\infty] + div(f_2)$.

So, $D = [P] - [Q] + n[\infty] + div(g_1)$ for some points $P, Q$ and some parameter $n$ and a rational function $g_1$.

sum($div(g_1)) = \infty$, it follows from sum($div(g)) = \infty$.

deg($D$) $ = 1 - 1 + n + 0 = n$
\begin{align*}
&D = [P] - [Q] + div(g_1) \\
&sum(D) = P - Q + sum(div(g_1)) = P - Q 
\end{align*}

Suppose $sum(D) = \infty$. Then $P - Q = \infty$, so $P = Q$ and $D = div(g_1)$.

Converse of the proof follows from the lemma below. 
\end{proof}

\begin{lemma}
    Let $P, Q \in E(\bar{K})$ and suppose there exists a function $h$ on $E$ with $div(h) = [P] - [Q]$. Then $P = Q$.
\end{lemma}

\begin{proof}
    Suppose $P \neq Q$ and $div(h) = [P] - [Q]$. Then, for any constant $c$, the function $h - c$ has a simple pole at $Q$. By \ref{zeros:poles}, it has exactly one zero.  

    Let $f$ be any function on $E$. If $f$ does not have a zero or pole at $Q$, then \\ $g(x, y) =\prod\limits_{R \in E(\bar{K})} {\left( h(x, y) - h(R) \right)^{ord_R (f)}}$ has the same divisor as $f$. 

    Each factor has a pole at $Q$ (of order $ord_R(f)$). Since $f$ and $g$ have the same divisor, the quotient $f/g$ has no zeros or poles, and therefore is constant. 

    Every function on $E(\bar{K})$ is a rational function on $h$ (in particular, $x$ and $y$ are rational functions of $h$), and the lemmas 11.5, 11.6 $^{\cite{Washington:book:2008}}$ show that such rational functions do not exist. 
    
    Hence, $P \neq Q$ is a contradiction. 
\end{proof}

\section{Construction of the Weil Pairing}
Let $n$ be an integer not divisible by $char(K)$ and $E$ be an elliptic curve over $K$.  We have already proved that $E[n] \subseteq E(K)$.

We want to construct a pairing $e_n : E[n] \times E[n] \rightarrow \mu_n$ where $\mu_n$ is the set of $n^{th}$ roots of unity in the algebraic closure of $K$. Using a result we have proved, we have $E[n] \subseteq E(K) \implies \mu_n \subset K$. 

\begin{enumerate} 
\item Let $T \in E[n]$. By \ref{div_existence}, $\exists f$ such that $div(f) = n[T] - n[\infty]$. 
\item The subgroup $E[n^2]$ contains every point that satisfies $n^2 P = \infty$, where $P \in E(\bar{K})$. So, $\exists \, T^{'} \in E[n^2]$ such that $nT^{'} = T$  since $n(nT^{'}) = nT = \infty$.
\item From \ref{div_existence}, there exists a function $g$ such that $div(g) = \sum\limits_{R \in E[n]} \left( [T^{'} + R] - [R] \right)$
\item Let $f \circ n$  be the function that multiplies the input $P$ with $n$ and then applies $f$ to it. Then $div(f \circ n) = n \left( \sum\limits_R [T^{'} + R] \right) - n \left( \sum\limits_R [R] \right) = div(g^n)$.
\item Let $S \in E[n]$ and let $P \in E(\bar{K})$. Then \\
$g(P+S)^n = f(n(P+S)) = f(nP) = g(P)^n$.
\item So, $\left(\frac{g(P+S)}{g(P)}\right)^n = 1$. Therefore,  $\left(\frac{g(P+S)}{g(P)}\right) \in \mu_n$.
\end{enumerate}

We can observe that $\frac{g(P+S)}{g(P)}$ is independent of point $P$. 
\begin{definition}
\label{weil_pairing}
The bilinear map $e_n : E[n] \times E[n] \rightarrow \mu_n$ is defined as follows : \\ For $S, T \in E[n]$, $e_n(S, T) = \frac{g(P+S)}{g(P)}$. 
\end{definition}

\newpage 
\section{MOV Attack}
The goal of this attack $^{\cite{MOV:1993}}$ on ECDLP is to reduce the given problem to an easier discrete logarithm problem. 
\begin{algorithm}
\caption{MOV attack on Discrete Logarithm problem}
\begin{algorithmic}[1]
\Procedure{MOV-attack}{P, Q}
\State Input : $P \in E(F_q)$ of order $N$
\State Output : $m$, where $Q = mP$ 
\State Determine smallest integer $k$ such that $E[n] \subseteq E(F_{q^k})$. 
\State Choose a random point $R \in E(F_{q^k})$ and compute the order M of R
\State Let $d = gcd(M, N)$, and let $R_1 = (\frac{M}{d}) R$.
\State Compute $u = e_N(P, R_1), v = e_N(Q, R_1)$
\State Compute $l$, the discrete logarithm of $v$ to the base $u$ in $F_{q^k}$. 
\State Repeat with random points R to get a congruence of the form $k (mod \,\, N)$. 
\EndProcedure 
\end{algorithmic}
\end{algorithm}

\subsection{Proof of correctness} 

The points of $E[n]$ have coordinates coming from the algebraic closure $\bar{F_p}$.  \\
From finite field theory, $\bar{F_p} = \bigcup\limits_{i = 1}^{\infty} F_{p^i}$ \\
Since there are a finite number of points in $E[n]$, we can choose $k$ to be the largest value of all such $i$'s. 
All the primitive $N^{th}$ roots of unity are contained in $F_{q^k}$. \\
From the bilinear properties of Weil Pairing, $e_N(Q, R_1) = e_N(mP, R_1) = {e_N(P, R_1)}^m$. \\
Now, we have an easier discrete logarithm problem in $F_{q^k}$ since $e_N(P, R_1) \in \mu_N$. 

